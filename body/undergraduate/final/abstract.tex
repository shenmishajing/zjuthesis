\cleardoublepage{}
\begin{center}
    \bfseries \zihao{3} 摘要
\end{center}

\par 宫颈癌是女性最常见的癌症之一,严重威胁着女性的生命和健康。每年全球都有数十万女性确诊宫颈癌,更有数十万女性因宫颈癌死亡。而宫颈癌是可以通过一些筛查手段在早期发现并予以治疗的,TCT细胞学检查就是一种常用的筛查手段。在TCT检查过程中,医生需要对宫颈细胞学图像进行分析诊断、寻找病变细胞,这一过程十分耗时费力,而且还面临着诊断具有主观性以及可能的漏检和误检等问题。为了帮助医生进行宫颈细胞学图像的分析和诊断,我们亟需计算机的辅助手段。现有的宫颈病变细胞检测方法基于自然图像上的目标检测算法,针对病变类别区分困难等问题提出了添加额外分类器等方法,这带来了一定的检测性能提升,但依然存在着单细胞与细胞簇存在形态差异、放大倍率不一致等问题。为了解决上述问题,本文提出了一种基于任务分解和半监督学习的方法来完成宫颈病变细胞检测的任务。基于任务分解,我们将图片中的标注框的每个类别分为单细胞与细胞簇两种,然后使用不同的网络模块分别完成这两种类别的识别。通过半监督学习,我们的网络在仅有少量正常细胞标注的情况下学会捕捉正常细胞的特征,并将捕获到的正常细胞的特征信息用于加强异常细胞的检测中的特征提取等过程。在合作医院提供的数据集上,我们的模型在检测的准确率上超越了Faster R-CNN等基线模型和其他现有的宫颈病变细胞检测方法。
\par 关键词:宫颈病变细胞检测,深度学习,任务分解,半监督学习,Faster R-CNN

\cleardoublepage{}
\begin{center}
    \bfseries \zihao{3} Abstract
\end{center}
\par Cervical cancer is one of the most common cancers in women, and it is a serious threat to women's life and health. Every year, hundreds of thousands of women worldwide are diagnosed with cervical cancer, and hundreds of thousands of women die from cervical cancer. Cervical cancer can be detected and treated early through some screening methods, and TCT cytology is a commonly used screening method. In the process of TCT examination, doctors need to analyze and diagnose cervical cytology images and find diseased cells. This process is very time-consuming and laborious, and it also faces problems such as subjective diagnosis and possible missed detection and misdetection. In order to help doctors analyze and diagnose cervical cytology images, we urgently need computer aids. The existing cervical lesion cell detection methods are based on the object detection algorithm on natural images, and methods such as adding additional classifiers are proposed for problems such as difficulty in distinguishing lesion types. This has brought a certain improvement in detection performance, but there are still problems such as morphological differences between single cells and cell clusters and inconsistent magnification. In order to solve the above problems, this paper proposes a method based on task decomposition and semi-supervised learning to complete the task of cervical lesion cell detection. Based on task decomposition, we divide each category of the label box in the picture into two types: single cell and cell cluster, and then use different network modules to complete the recognition of these two categories. Through semi-supervised learning, our network learns to capture the characteristics of normal cells with only a small number of normal cell annotations, and uses the captured characteristic information of normal cells to strengthen the process of feature extraction in the detection of abnormal cells. On the data set provided by the partner hospital, our model surpasses baseline models such as Faster R-CNN and other existing cervical lesion cell detection methods in terms of detection accuracy.
\par Keyword: Detection of cervical lesion cells, deep learning, task decomposition, semi-supervised learning, Faster R-CNN
