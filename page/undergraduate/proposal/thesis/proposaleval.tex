\cleardoublepage

\ifthenelse{\equal{\MajorFormat}{cs}}
{
    \chapternonum{毕业论文(设计)文献综述和开题报告考核}
    \bfseries

    {
        \zihao{4}
        \noindent 导师对开题报告、外文翻译和文献综述的评语及成绩评定:
    }
    \par 两癌筛查是国家重点筛查项目,其中对宫颈TCT异常细胞检测是关键核心步骤,请调研相关研究工作,并在State-of-art工作基础上优化模型。


    \vspace{50mm}
    \thesisproposaleval[8][13][4]
    \signature{导师签名}

    {
        \zihao{4}
        \noindent 学院盲审专家对开题报告、外文翻译和文献综述的评语及成绩评定:
    }
    \par 开题报告格式较为完整,内容较为详实,提出利用半监督学习和任务分解方法用于宫颈病变细胞检测。外文翻译较为清晰,但没有达到“要与基于半监督学习和任务分解的宫颈病变细胞检测有关”的要求,文献综述内容较为详实。


    \mbox{} \vfill
    \thesisproposaleval[8][13][4]
    \signature{开题报告审核负责人(签名/签章)}
}
{
    \chapternonum{毕业论文(设计)文献综述和开题报告考核}
    \bfseries

    {
        \zihao{4}
        \noindent 对文献综述、外文翻译和开题报告评语及成绩评定:
    }


    \mbox{} \vfill
    \thesisproposaleval
    \signature{开题报告答辩小组负责人(签名)}
}
